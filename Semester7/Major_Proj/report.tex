\documentclass[14pt]{extarticle}

\usepackage[english]{babel}
\usepackage[utf8]{inputenc}
\usepackage{hyperref}
\usepackage{graphicx,eso-pic}
\usepackage{newtxtext}
\usepackage{setspace}
\usepackage{lipsum}
\usepackage{multicol}
\usepackage{titlesec}
\usepackage{pdfpages}
\usepackage{indentfirst}
\usepackage[bottom=1.5cm,top=2.5cm,left=2cm,right=2cm]{geometry}

\hypersetup{
    colorlinks=true,
    linkcolor=black,
    filecolor=magenta,
    urlcolor=blue,
    pdftitle={Project Report},
    pdfpagemode=FullScreen,
    }

\urlstyle{same}

\makeatletter

\setlength{\parskip}{1em}

\newcommand\frontmatter{
    \cleardoublepage
    \pagenumbering{roman}
}

\newcommand\mainmatter{
    \cleardoublepage
    \pagenumbering{arabic}
}

\newcommand\backmatter{
    \if @openright
        \cleardoublepage
    \else
        \clearpage
    \fi
}

\makeatother

\titleformat{\section}[block]{\Large\bfseries\filcenter}{}{1em}{}


\vspace{-3em}

\title{Educational Chat-Bot\\
Using Artificial Intelligence}
\author{}
\date{}

\begin{document}

\frontmatter

\newgeometry{bottom=2cm,top=2cm,left=1.5cm,right=1.5cm}
\maketitle

\vspace{-7em}

\begin{center}
    \singlespacing
\textbf {A PROJECT REPORT} \\
\emph {Submitted by} \\
\textbf {RISHABH ANAND} \\
\textbf {19BCS4525 }   \\


\vspace{1.5em }
Under the Supervision of:\\
Aanchal Sharma \\

\singlespacing

BACHELOR OF ENGINEERING \\
IN \\
COMPUTER SCIENCE and ENGINEERING\\
(Internet of Things)

\vspace{1em}
\includegraphics[scale=0.3]{private/banner.png}

\singlespacing

APEX INSTITUTE OF TECHNOLOGY\\
CHANDIGARH UNIVERSITY, GHARUAN\\
Mohali, Punjab \\

\vspace{2em}
\onehalfspacing
October, 2022

\end{center}
\restoregeometry

\newpage
\begin{center}
    \includegraphics[scale=0.34]{private/banner.png}
\end{center}
\section*{Bonafide Certificate}

Certified that this project report "Educational Chat-Bot Using AI" is the bonafide work of "Rishabh Anand" who carried out the project work under my supervision.

\begin{multicols*}{2}
\begin{center}
\vspace*{5em}
\textbf{SIGNATURE}\\
\phantom{ }\\
\emph{Name}\\
\phantom{ }\\
HEAD OF THE DEPARTMENT\\
\phantom{ }\\
\emph{Department}\\
\vspace{3em}
Submitted for the project viva-voce examination held on\\
\vspace{5em}
\textbf{INTERNAL EXAMINER}\\
\vspace{10em}
\phantom{ }\\
\vspace{6em}
\textbf{SIGNATURE}\\
\phantom{ }\\
\emph{Aanchal Sharma}\\
SUPERVISOR\\
Academic Designation\\
\phantom{ }\\
\emph{Department}\\
\vspace{9em}
\phantom{}\\
\textbf{EXTERNAL EXAMINER}\\
\end{center}
\end{multicols*}

\newpage
\section*{Acknowlgement}

I, 'Rishabh Anand' student of 'Bachelor of Engineering in Computer Science and Engineering - IoT',
session: 2019-23, Department of Computer Science and Engineering, Apex Institute of Technology, Chandigarh University,Punjab,
hereby declare that the work presented in this Project Work entitled 'AI Chat-Bot'
is the outcome of my own bona fide work and is correct to the best of our knowledge
and this work has been undertaken taking care of Engineering Ethics. \\

It contains no material previously published or written by another person
nor material which has been accepted for the award of any other degree or diploma of the university or other institute of higher learning,
except where due acknowledgment has been made in the text.

\vspace{5em}
\begin{flushright}
    Rishabh Anand\\
    19BCS4525
\end{flushright}

\vspace{13em}
Date: 06th October, 2022 \\

Place: Ludhiana (Punjab)\\


\newpage
\begin{center}
    \tableofcontents
\end{center}

\newpage
\addcontentsline{toc}{section}{List of Figures}
\listoffigures

\newpage
\addcontentsline{toc}{section}{ABSTRACT}
\section*{ABSTRACT}
\onehalfspacing
\setlength{\parskip}{0em}

In a world, ruled by knowledge and one that runs on information, it becomes crucial for a person to know what is happening around him. But with the abundance of knowledge and information it becomes difficult for a person to keep his own data bank up to date with all the facts flowing around him. \\

This is where my project comes in. Every piece of knowledge, every bit of information just one question away. Not only that, but it also helps you to keep your data bank up to date with the latest information. Sure you could just google stuff and get that required bit of data but google is big and the results are crowded. How do you fiter out the important parts of the information that is relevant to you ? How to you personalize it to the area or to the institution that you are in ? With my project, of course. Not only are the results about "your" institute, but it also comes with the relevance that you didn't know you needed. \\

The results are summarized, directly from your own instutute's webpages. The results are personalized to your needs. The results come in a conversational manner. It's like talking to the dean of the college but in a far more friendly and fun way. So now you won't have to go to the poorly designed college webpage. All you need to do is talk to this bot, like you would talk to one of your friends and all the information that you need is given to you. Instantly.\\

This project solves the problem of FOMO in an elegant way. Thereby, making the world, or at least your world, a better place.

\mainmatter

\setlength{\parskip}{1em}

\newpage
\section{INTRODUCTION}

\subsection{Need Identification}

A number of students miss out on important information that is being circulated within an institution if they are not paying attention or miss out on classes.

\subsection{Problem Identification}

Missing out on this information causes a lot of issues for them such as missing:: \\
\vspace*{-3em}
\begin{itemize}
    \item Placement drives
    \item Events
    \item Tests
    \item Submissions
\end{itemize}

\subsection{Tasks Identification}

The major tasks include ::
\vspace*{-1.5em}
\begin{itemize}
    \item Generating a data-bank using institute's site
    \item Generating a data-bank using institute's noticeboards
    \item Regularly updating the data-bank
    \item Providing a user-friendly way to convey this information to the relevant people.
    \item Providing a way to personalize the information to the user.
\end{itemize}

\subsection{Timeline}

\begin{figure}[!htb]
    \begin{center}
        \includegraphics[width=\textwidth]{private/Timeline.png}
    \end{center}
    \caption{Timeline}
\end{figure}

\subsection{Organization of Report}

\begin{enumerate}
    \item Literature Survey :: Includes information about similar previousp projects.
    \item Design Flow :: Discusses the design decisions taken during the project building.
    \item Result Analysis :: Discusses the methodology adopted for the project.
    \item Conclusion :: Discusses the conclusion of the project and its future aspect.
\end{enumerate}


\newpage
\section{LITERATURE SURVEY}

\subsection{Timeline of the reported problem}

The problem starts as soon as we join any institution as that is when most of the important information is passed around that we have to keep a track of. With the abundance of information comes ignorance of knowledge. The real problem starts once we reach our final semesters and it's the time of placements. There are mulltiple companies coming to the campus everyday and conductiong events every hour. In all this chaos, it is hard to keep track of the companies that we might be interested in or the companies that actually are good for our career depending upon the role that we desire. 

\subsection{Proposed Solutions}

There have been attempts to address this problem before as well, from both the institution as well as the students to make it easier to keep track of things and news.  \\

A notice board : where institutions update the students with the news. A low graphic, UX ignorant dashboard with the entireity of information thrown right at your face. Trusted and updated but neither elegant nor effective. \\
Emails : where institutions update the student with the news and placemnet oppourtinities. A slow, non reliable source of information as many student accidentally put the mail in spam and don't recieve from the sender anymore. \\  

The idea that students came up with is mobile applications. Where they can keep track of information. The news around the campus. This however doesn't cover the emails. It is also not reliable you still need a seperate app in your phone just for this thing.

\subsection{Review Summary}

From the above solutions we have two scenarios, one where the information is available but not presented properly which makes it hard to convey the information. The other one where neither the information is complete nor is it easily accessible and isolated. We also saw that we need to tackle with different data sources and present information in a condensed and private manner where the information is personalized for the user and relevant to him. We also saw that trhe solution needs to be platform independent to make it easily accessible.
\subsection{Problem Definition}

Provide information related to campus activities and placements by compiling different data sources and presenting the data in a elegant and effective layout which is personalized and relevant to different users.
\subsection{Goals}

\begin{itemize}
    \item Different Data Sources (Notcie-Board and Emails) Compiled.
    \item Filtering of Information.
    \item Personalization of Information.
    \item Good UX and Content.
    \item Cross-Platform Solution.
\end{itemize}

\newpage
\section{DESIGN FLOW}

\subsection{Selection of Features}
\subsection{Design COnstraints}
\subsection{Feature Analysis}
\subsection{Design Flow}
\subsection{Design Selection}
\subsection{Implementation Plan}


\newpage
\section{RESULTS ANALYSIS AND VALIDATION}

\subsection{Implementation}



\newpage
\section{CONCLUSIONS}

\subsection{Conclusion}
\subsection{Future Work}


\newpage
\addcontentsline{toc}{section}{REFERENCES}
\section*{REFERENCES}


\newpage
\addcontentsline{toc}{section}{APPENDIX}
\section*{APPENDIX}


\newpage
\addcontentsline{toc}{section}{USER MANAUAL}
\section*{USER MANUAL}

\end{document}